\chapter{Summary and conclusion}
This thesis presents a search for the excited states of light u,d and heavy b flavor quarks in the $\gamjet$ and $\gambjet$ final states respectively, using the
pp collision data at $\sqrt{s}$ $=$ 13\unit{TeV}, collected by the CMS detector in 2016. This data correspond to a total integrated luminosity of
35.9\unit{\fbinv}.

The underlying processes which have been studied are pp $\rightarrow$ \qstar $\rightarrow$ $\gamjet$ and
pp $\rightarrow$ \bstar $\rightarrow$ $\gambjet$. The resonance signals for \qstar and \bstar, if exist, will show their presence in the form of a bump
over the continuous invariant mass distribution spectrum of $\gamjet$ and $\gambjet$ respectively. Therefore, the invariant mass distributions are the
most important distribution for this study. The signal regions in the mass range from 1\unit{TeV} to 9\unit{TeV} for \qstar and from 1\unit{TeV}
to 5\unit{TeV} for \bstar with a small gap of 50\unit{GeV}, have been scanned for the presence of any expected signature of excited quarks. Variable mass binning with
bin width comparable to the signal mass resolution has been considered for the invariant mass spectra. The event statistics for \qstar
search has been formed by requiring a high quality photon and a high quality jet.
The kinematical restrictions on the selected photons and jets are put in order to optimize the signal selection while reducing the background. In order to form event
statistics for \bstar search, the b-flavor of selected jets has been checked using a b-tag discriminator. The fraction of jets passing the discriminator form
the 1 b-tag category while the rest failing the discriminator form the 0 b-tag category. The likelihoods of two categories have been combined together in \bstar
search to avoid any statistical power loss due to the event migration between the two categories. The data events are corrected for pile-up effects and
various object energy corrections are applied. The event selection has been optimized for different analysis cuts in order to obtain the best signal
discrimination. The highest mass event in data is observed at M$_{\gamjet}$ $=$ 4.6\unit{TeV}. 

This analysis has two main backgrounds coming from SM $\gamjet$ and QCD dijet process. The MC samples corresponding to these backgrounds are used to validate
the data as per the SM expectations. At every stage of this analysis, the data are found to be in good agreement with the SM background predictions.  
However, the actual background for this search has been obtained by fitting the data with a smooth parameterization.
A significance test has been performed in order to look for any significant deviation in the data compared to the SM background expectations.
The most significant deviation observed is at 2.6$\sigma$, which can be safely considered as a statistical fluctuation. 

In the absence of any expected signal, upper limits on the cross-sections or lower limits on the masses of \qstar and \bstar have been set at the 95$\%$ confidence level
using the ``Higgs COMBINE Tool'' with the frequentist approach in the asymptotic approximation. The cross-section upper limits are compared with the
theoretical predictions in order to set the lower mass limits on the excited states of quarks. The excited light quarks within the mass range
1.0 $<$ M$_{\qstar}$ $<$ 5.5\unit{TeV} and excited b-quarks within the mass range 1.0 $<$ M$_{\bstar}$ $<$ 1.8\unit{TeV} are excluded at the 95$\%$ confidence level,
corresponding to standard model like coupling strengths ($f$ $=$ 1.0).

\textit{\textbf{The limits presented in this thesis are the most
  sensitive limits for \qstar and \bstar searches in the ${\gamma}+$jet and ${\gamma}+$b-jet final states.
  The search for the excited b-quarks has been presented for the first time
  in any final state at $\sqrt{s}$ $=$ 13 TeV.}}

The results of this study have been accepted for publication in Physics Letters B (Ref.\ No.\: PLB$-$D$-$17$-$01732R1). 



