\begin{center}
  {\large \textbf{Search for Quark Compositeness in $\mathbold{\gamma}+$jet final states in proton-proton collisions at $\mathbold{\sqrt{s}}$ $=$ 13 TeV with the CMS Detector at the Large Hadron Collider}} \\
%
%\vspace{0.5cm}
%
%{\large \textbf{Rocky Bala Garg}} \\
%
%{\large \textbf{Department of Physics \& Astrophysics \\ University of Delhi}} \\
%
\vspace{0.4cm}
%
%%{\large \emph{THESIS ABSTRACT}}
{\large {ABSTRACT}}
%\doublespacing
%{\Large \textbf{ ABSTRACT}}
%\vspace{0.5cm}
\end{center}
\onehalfspacing
The main aim of the particle physics is to investigate the fundamental building blocks of matter and the forces governing them. Within the
context of the standard model of particle physics, quarks and leptons are considered to be fundamental. Their existence have been verified
by a number of experiments and their properties are extensively studied. In spite of its great success, the standard model is not considered as a
complete theory. There are a number of unexplained phenomena which predict the existence of some new physics beyond the standard model. The existence of three
nearly identical generations of quarks and leptons strongly suggest the possibility of an underlying sub-structure. A variety of models, referred to as,
the compositeness models, are theorized which consider the quarks and leptons to be made up of more fundamental particles, known as ``preons''. A convincing
signature of the quarks sub-structure is the observation of their excited states. These excited quarks are supposed to interact with
their standard model counterparts through gauge interaction for the compositeness scale, $\Lambda\le\sqrt{s}$. Various searches have been performed in the
past to observe excited states but no evidence has been found. 

This thesis presents a search for the excited states of light u, d and heavy b flavor quarks in the $\gamjet$ and $\gambjet$ final states respectively, using the
pp collision data at $\sqrt{s}$ $=$ 13\unit{TeV}, collected by the CMS detector in 2016. This data correspond to a total integrated luminosity of
35.9\unit{\fbinv}. The excited quark signals, if exist, will show their presence in the form of bumps over the continuous invariant mass distribution spectrums
of $\gamjet$ and $\gambjet$ states respectively. Many standard model processes can also imitate these final states, thereby, forming
the backgrounds of this study. This analysis has two main backgrounds coming from SM $\gamjet$ and QCD dijet processes, with a small contribution from W$/$Z $+$ jet
processes. The MC samples corresponding to these backgrounds are used to validate the data as per the standard model expectations.
The actual background of this study has been
obtained by using a data driven technique. The excited quarks signals are generated and simulated using the event generator \pythia to scan a broad mass range.

In the absence of any expected signal, upper limits on the cross-sections or lower limits on the masses of excited quarks have been set at the 95$\%$ confidence level
using the frequentist approach in the asymptotic approximation. A comparison of cross-section upper limits has been made with the theoretical predictions of
excited quarks in order to set the lower mass limits on the excited states. The excited light quarks within the mass range
1.0 $<$ M$_{\qstar}$ $<$ 5.5\unit{TeV} and excited b-quarks within the mass range 1.0 $<$ M$_{\bstar}$ $<$ 1.8\unit{TeV} are excluded at the 95$\%$ confidence level,
corresponding to standard model like coupling strengths. The limits presented in this thesis are the most
sensitive limits for excited quark searches in the $\gamjet$ and $\gambjet$ final states. The search for the excited b-quarks has been presented for the first time
in any final state at $\sqrt{s}$ $=$ 13\unit{TeV}.

The work presented in this thesis has been accepted for publication in Physics Letters B (Ref.\ No.\: PLB$-$D$-$17$-$01732R1).



