\documentclass[12pt,a4]{article}

\usepackage{hyperref}
\usepackage[T1]{fontenc}
\usepackage{microtype}
%% packages for table styles
\usepackage{booktabs,tabularx,color,colortbl, xfrac}
\usepackage{graphicx}
%% Equations alignment
\usepackage[fleqn]{amsmath} %% left
\usepackage{amsfonts,MnSymbol}
\usepackage{amsthm}
\usepackage{cite}
\usepackage{subfig}
\usepackage{float}
\usepackage{rotating}
\usepackage{bm} % for bold math font
\usepackage{multirow}
\usepackage{verbatim}
\usepackage{xspace}
\usepackage{caption}
\usepackage{cleveref}
\usepackage{fixmath}
\usepackage{array}
\usepackage{wallpaper}
\usepackage{epstopdf}
\usepackage{notoccite} %% To avoid citations in List of figures
\usepackage{soul} %% to underline without wraping text
\usepackage{url}
\usepackage[utf8]{inputenc}
\usepackage{graphicx}
\usepackage[left=1.5in, right=1.5in, top=1.25in, bottom=1.25in]{geometry}

\usepackage{amsmath}
\usepackage{amssymb}
\usepackage{amsthm} 
\usepackage{setspace}
\usepackage{rotating}
\usepackage{tikz}
\usepackage{caption}
\usepackage{mathrsfs}
\usepackage{pstricks}
\usepackage{fixmath}
\usepackage{color}
\usepackage{times}
\usepackage{afterpage}
\usepackage{longtable}
\usepackage{cancel}  % for MET
\usepackage{verbatim}
% Counter commands
\setcounter{page}{1}
\setcounter{secnumdepth}{4}
\setcounter{tocdepth}{3}
%\setcounter{lofdepth}{2}
\newcommand{\bra}[1]{\ensuremath{\left\langle#1\right|}}
\newcommand{\ket}[1]{\ensuremath{\left|#1\right\rangle}}

\usepackage{symbolDefn}

\begin{document}

\pagenumbering{arabic}

\pagestyle{plain}

\thispagestyle{empty}

\begin{center}

{\Large \textbf{Search for Quark Compositeness in $\mathbold{\gamma}$ $+$ jet final states in proton-proton collisions at $\mathbold{\sqrt{s}}$ = 13 TeV with the CMS Detector at the Large Hadron Collider}}
\vspace{1cm}

\noindent{\large{A SYNOPSIS FOR THE DEGREE OF\\
    \textbf{DOCTOR OF PHILOSOPHY}}}\\

\vspace{8mm}
\begin{figure}[h]
\centering
\includegraphics[width=4cm]{DU_logo.png}
\end{figure}

\vspace{13mm}
\noindent\uppercase{\large{\vspace{2cm} \textbf{Rocky Bala Garg}\\}}
\vspace{15mm}
\uppercase{\large{\textbf{Prof. Brajesh C. Choudhary} \hspace{0.9cm} \textbf{Prof. Sanjay Jain}}}\\
\uppercase{\hspace{2.2cm}\small{Supervisor \hspace{4.3cm} Head of the Department}}\\

\vspace{10mm}

\noindent\uppercase{{Department of Physics \& Astrophysics}}\\
\noindent\uppercase{{University of Delhi}}\\
\noindent\uppercase{{Delhi - 110 007\\ India}}\\
\vspace{5mm}
\noindent\uppercase{ \textbf{2018}}\\ 
\end{center}

\thispagestyle{empty}

%% ABSTRACT
\begin{center}
  \clearpage
  \vspace{5mm}
  \noindent\uppercase{\Large \textbf{Abstract}}
  \vspace{1cm}
\end{center}
Quarks and leptons are considered as fundamental particles within the context of the standard model of particle physics. However, a number of unexplained phenomena
incorporated within the standard model predict the existence of some new physics beyond the standard model. The existence of three generations of quarks and
leptons lead to the
possibility of an underlying sub-structure. A variety of models, referred to as, the compositeness models, consider the quarks and leptons to be made up of
more fundamental particles, known as ``preons''. A convincing signature of the quarks sub-structure is the observation of their excited states.
These excited quarks are supposed to interact with their standard model counterparts through gauge interactions for the compositeness scale, $\Lambda\le\sqrt{s}$.
Various searches have been performed but no evidence of excited states has been found yet. 

This thesis presents a search for the excited states of light u, d and heavy b flavor quarks, respectively, in the $\gamjet$ and $\gambjet$ final states using the
pp collision data at $\sqrt{s}$ $=$ 13\unit{TeV} collected by the CMS detector during 2016. This data correspond to a total integrated luminosity of
35.9\unit{\fbinv}. The excited quark signals, if exist, will show their presence in the form of bumps over the continuous invariant mass distribution spectra 
of $\gamjet$ and $\gambjet$ states respectively. Many standard model processes can also imitate these final states, thereby, forming
the backgrounds for this study. This analysis has two main backgrounds coming from SM $\gamjet$ and QCD dijet processes, with a small contribution from W$/$Z $+$ jet
processes. The MC samples corresponding to these backgrounds are used to validate the data as per the standard model expectations.
The actual background of this study has been obtained by using a data driven technique. The excited quarks signals are generated and simulated using
the event generator \pythia to scan a broad mass range.

In the absence of any expected signal, upper limits on the cross-sections or lower limits on the masses of excited quarks have been set at the 95$\%$ confidence level
using the frequentist approach in the asymptotic approximation. A comparison of cross-section upper limits has been made with the theoretical predictions of
excited quarks in order to set the lower mass limits on the excited states. The excited light quarks within the mass range
1.0 $<$ M$_{\qstar}$ $<$ 5.5\unit{TeV} and excited b-quarks within the mass range 1.0 $<$ M$_{\bstar}$ $<$ 1.8\unit{TeV} are excluded at 95$\%$ confidence level,
corresponding to standard model like coupling strengths. The limits presented in this thesis are the most
sensitive limits for excited quark searches in the $\gamjet$ and $\gambjet$ final states. The search for the excited b-quarks has been presented for the first time
in any final state at $\sqrt{s}$ $=$ 13\unit{TeV}.

%% SYNOPSIS
\begin{center}
  \clearpage
  \vspace{5mm}
  \noindent\uppercase{\Large \textbf{\underline{SYNOPSIS}}}
  \vspace{1cm}
\end{center}
\onehalfspacing
Particle physics is a branch of physics that tries to answer the fundamental questions such as, what the world is made of and how it came into existence?
It allows us to explore the
frontiers and boundaries of the universe at the levels unmatched by any other field. It is also termed as ``High Energy Physics'' because it requires microscopic
particles to interact at tremendously high energies to observe new and interesting phenomenon. The theoretical model that defines the properties of fundamental
particles and their interactions is known as ``Standard Model'' (SM)~\cite{Glashow:1961tr, Salam:1964ry, Weinberg:1967tq,Gross:1973id, Politzer:1973fx}.
The standard model incorporates three of the four fundamental interactions, namely electromagnetic, weak and strong, as well as classify all the known elementary
particles. It is a quantum field theory based on a complex mathematical formulation which encodes all the information in the form of a mathematical description,
known as ``Lagrangian''. Over the years, standard model has been able to explain a number of experimental evidences within its framework. Its most significant
achievement is the prediction of the Higgs boson, discovered by the CMS and ATLAS experiments in 2012.
Even after its huge success, the standard model can not be considered as a complete theory of nature. The biggest reason being the non-inclusion of gravity within its
framework. It also describes only 5$\%$ of the universe, with no explanation for the rest of 95$\%$, which is made up of dark matter and dark energy.
It also does not have any explanation for baryon asymmetry, neutrino oscillations and many other such phenomena. 

The descriptions for such unexplained phenomena can only be achieved by theories which go beyond the standard model (BSM) of particle physics.
A number of BSM theories have been proposed \eg supersymmetry, technicolor, extra-dimensions, compositeness \etc, to name a few.
The study represented in this thesis is based on a model that considers the excited state of quarks ($\qstar$) and leptons ($\lstar$),
also known as the compositeness models~\cite{Pati:1975md, Eichten:1983hw, Baur:1987ga, Baur:1989kv}.
The prime motivation for these models are the presence of three quark and lepton generations with similar properties. This replication points toward
some underlying structure identical in all the families. The most implicit signature of this sub-structure is the observation of a fermion (quark or lepton)
in the excited state. The excited states behave as the resonance states which then eventually decay into SM particles. 

The fundamental particles that quarks and leptons are considered to be composed of, are known as ``preons''. These preons are postulated to experience a new kind of
force that becomes very strong at a particular scale $\Lambda$, known as the compositeness scale, forming quarks and leptons bound states. 
Excited quarks and leptons can interact with their ordinary counterparts through contact interactions for $\Lambda$ $>>$ $\sqrt{s}$ and through gauge
mediation for $\Lambda$ $<$ $\sqrt{s}$, where $\sqrt{s}$ is the center of mass energy. The model used in this study assumes compositeness scale, $\Lambda$
$<$ $\sqrt{s}$, and mass scale of excited quarks $M_{\qstar}$ = $\Lambda$. In this regime, the gauge interactions dominate over contact interactions.
This model~\cite{Baur:1989kv} considers a $SU(3)$ $\times$ $SU(2)$ $\times$ $U(1)$ symmetry with ground state quarks in the
form of left-handed doublets and right-handed singlets while the excited state quarks in the form of left- and right-handed doublets. 
The gauge interaction mediated between ordinary and excited quarks by gauge bosons is obtained through the
requirement of gauge invariance and is of magnetic-moment type, defined by the effective Lagrangian:
\begin{equation}
{\mathcal L}_{int} = \frac{1}{2\,\Lambda}\bar{q^{\ast}_{R}} \, \sigma^{\mu\nu}
\left[g_{s}f_{s}\frac{\lambda^{a}}{2}G^{a}_{\mu\nu}\;+\;gf\frac{\tau^{b}}{2}W_{\mu\nu}^{b}\;+\;g'f'\frac{Y}{2}B_{\mu\nu} \right] q_{L} + h.c.,
\label{eq:Lagrangian1}
\end{equation}
Here $G^{a}_{\mu\nu}$, $W_{\mu\nu}$, and $B_{\mu\nu}$ are the field-strength tensors for $SU(3)$, $SU(2)$, and $U(1)$ interactions, while $\lambda_{a}$, $\tau_{b}$, $Y$
are the corresponding generators. The constants $f_{s}$, $f$ and $f'$ determine the strength of a coupling and are
evaluated using the compositeness dynamics. These are considered to be equal to unity to allow for the standard model like coupling strengths. 
A composite quark can acquire an excited state by the absorption of a gluon and will radiate a photon, gluon or weak boson on returning to the ground state.
This study considers an excited light ($\ustar,\dstar$) and heavy ($\bstar$) flavor quark decaying into a quark and a photon final state.

In this thesis, a complete analysis of this final state has been presented to search for any possible signature of excited quarks.
The thesis has been organized in five chapters. A brief description of each chapter is provided below. 

\textbf{Chapter one} describes the evolution of particle physics and highlights the important milestone discoveries as well as
the remarkable contributions from various scientists. It further provides an overview of the underlying theoretical framework of the standard model of particle physics,
illustrating its three landmark theories: Quantum Electrodynamics, Electroweak unification
and Quantum Chromodynamics, as well as the Higgs mechanism~\cite{Higgs:1964ia, Englert:1964et, Higgs:1964pj, Guralnik:1964eu, Higgs:1966ev}.
It lists the various shortcomings of standard model and the requirements for a more versatile theory. In the end, this chapter also gives a
broad idea of the excited quark model used in this study.

\textbf{Chapter two} discusses about the experimental apparatus used to accumulate the data to perform this important study. A detailed description of
the design and performance of the ``Large Hadron Collider'' (LHC)~\cite{Bruning:782076, Evans:2008zzb} accelerator
and ``Compact Muon Solenoid'' (CMS)~\cite{cmsTDR} detector, housed at the
``European Organization for Nuclear Research'' (CERN) on the Swiss-French border in Geneva, Switzerland, has been provided. The Big-Bang theory explaining the origin
of the universe and the usefulness of LHC in re-creating the similar situations in the laboratory has been explained. The CERN accelerator complex~\cite{Web:CERN}
consists of a number of
accelerating facilities deployed to provide acceleration to the proton beam. A list of the important parameters of LHC has been provided in the form of a table.
A detailed explanation of the CMS detector along with its various sub-detectors, the main experimental challenges faced by it, the co-ordinate system used,
\etc has been provided. A section explaining the CMS trigger and data acquisition system~\cite{triggerTDR, daqhltTDR}
as well as the process of CMS data management has also been documented. 

\textbf{Chapter three} took up the discussions of event reconstruction~\cite{Lange:2011zza} and event simulation~\cite{robert2004monte} processes at CMS.
The detector provides information about the collisions
and the undergoing physics processes in the form of electronic signals only. A number of steps are to be followed to obtain a meaningful physics result from these
raw information.
The RAW data in the form of digitized electronic signals is made to pass through a software based multi-level process which results into the events reconstruction.
For the purpose of calibration and comparison, the behaviour of particles within the detector is also simulated by the use of event generators.
This chapter also details the different steps and procedures used in the reconstruction and simulation of the events for CMS physics analyses. The event reconstruction
step considers the track and vertex reconstruction as well as the various objects reconstruction and identification. The event simulation step considers the event
generation as well as the detector simulation.

\textbf{Chapter four} reports the various steps and procedures followed in the analysis of the CMS data to look for any signatures of excited light- and heavy-quark
presence in $\gamjet$ final states. The analysis strategy emphasizing the importance of invariant mass distribution spectrum of $\gamjet$ system has been documented.
This chapter also presents the detailed overview of the signals and backgrounds for this study,
along with the data and MC samples used. The event selection process to select the events
with a good photon and a good jet per event has been given in great detail. Based on the event selection, three different categories for \qstar and \bstar search have been
formed. A trigger requirement of a high pt photon is implemented to select an initial set of events. The data events are also subjected to some noise cleaning
filters. The officially recommended identification criteria has been used to select photons and jets. For \bstar search, the jets are also filtered through
b-tagging algorithms. The different selections used in this analysis are optimized to obtain best expected resonance mass limits.
The data sample consists of a $\gamjet$ event
with a highest invariant mass of 4.6\unit{TeV}. The background for this study has been estimated using a data driven technique which makes use of a polynomial function
fit with the data. The frequentist approach in asymptotic approximation has been used to determine the mass limits on \qstar and \bstar resonances. The final results
are presented in the form of significance of largest fluctuations in data and the lower mass limits on \qstar and \bstar resonances. 

\textbf{Chapter five} provides the summary and conclusions for this thesis. The final mass limits on the \qstar and \bstar resonances have been
documented and these limits are reported to be the best limits obtained so far
in $\gamjet$ final state. The results of this analysis have been accepted for publication in Physics Letters B (Ref.\ No.\: PLB$-$D$-$17$-$01732R1). 

\newpage
\bibliographystyle{CMSPubComm}
\bibliography{Synopsis_ref}

\newpage
\begin{center}
  \textbf{\large{\underline{List of Publications}}}
  \begin{itemize}
  \item \textbf{Search for excited quarks of light and heavy flavor in $\mathbold{\gamma}$ $+$ jet final states in proton-proton collisions at $\mathbold{\sqrt{s}}$
    = 13 TeV}
    \begin{itemize}
    \item Authors: CMS Collaboration.
    \item Journal: Accepted for publication in Physics Letters B (Ref.\ No.\: PLB$-$D$-$17$-$01732R1).
    \item Impact Factor: 4.8.
    %\item DOI: 
    %\item Journal Reference: 
    \item ArXiv id: arXiv:1711.04652.
      \item CMS Analysis Note: CMS-AN-16-462: Brajesh C. Choudhary, Rocky Bala Garg, Varun Sharma.
    \end{itemize}
  \item \textbf{Search for excited states of light and heavy flavor quarks in the $\mathbold{\gamma}$ $+$ jet final state in proton-proton collisions
    at $\mathbold{\sqrt{s}}$ = 13 TeV}
    \begin{itemize}
    \item Authors: CMS Collaboration.
    \item Physics Analysis Summary: CMS-PAS-EXO-17-002.
    \item ArXiv id: arXiv:1711.04652.
    \item CMS Analysis Note: CMS-AN-16-462: Brajesh C. Choudhary, Rocky Bala Garg, Varun Sharma.
    \end{itemize}
  \item \textbf{Search for excited quarks in $\mathbold{\gamma}$ $+$ jet final state in proton-proton collisions at $\mathbold{\sqrt{s}}$ = 13 TeV}
    \begin{itemize}
    \item Authors: CMS Collaboration.
    \item Physics Analysis Summary: CMS-PAS-EXO-16-015.
    \item CDS entry: http://cds.cern.ch/record/2204915.
      \item CMS Analysis Note: CMS-AN-15-262: Brajesh C. Choudhary, Rocky Bala Garg, Varun Sharma.
    \end{itemize}
  \end{itemize}
\end{center}
\pagebreak
\begin{center}
  \textbf{\large{\underline{Participation in Conferences / Workshops / Schools}}}
  \begin{itemize}
  \item Presented a poster titled \textbf{``Quark substructure at CMS''} at the XXII DAE$-$BRNS High Energy Physics Symposium held at the University of Delhi,
    December 12$-$16, 2016.
  %\item \textbf{Exotica general meeting during approval for EPS17}: Presented approval talk for excited light and heavy flavor quark analysis on June 29, 2017.
  %\item \textbf{Exotica general meeting}: Presented Pre-approval talk for excited light and heavy flavor quark analysis on April 5, 2017.
  %\item \textbf{Exotica Non-Hadronic subgroup meeting}: Regularly presented the status of excited light and heavy flavor quark analysis at center of mass energy of 13 TeV.
  %\item \textbf{Xsec task force meeting}: Presented a talk on ``k-factor calculation for $\gamma$ $+$ jet sample using MCFM software'', on February 15, 2016.
  \item Attended a symposium on \textbf{``Particle physics at the crossroads''}, organized by the University of Delhi and University of Edinburgh
    and held at the India International Center, New Delhi, India, February 15$-$17, 2013.
  \item Attended workshop on \textbf{``Whats next at LHC''} at the Tata Institute of Fundamental Research, Mumbai, India, January 6$-$8, 2014.
  \item Attended \textbf{``IX$^{\textrm{th}}$ SERC School on Experimental High Energy Physics''} held at the Department of Physics, IIT Madras, Chennai, India,
    December 2$-$21, 2013.
  %\item Attended \textbf{``CMS Data Analysis School''} held at Saha Institute of Nuclear Physics, Kolkata, India (November 7 - 11, 2013).
  %\item Attended workshop on \textbf{``Information Literacy and Competency''} held at the University of Delhi, India (January 17 - 18, 2013).
  \item Attended \textbf{``XXIV SERC Preparatory School on Theoretical High Energy Physics''} held at the Department of Physics, University of North Bengal,
    Darjeeling, West Bengal, India,  September 10$-$October 8, 2012.
  \end{itemize}
\end{center}
\pagebreak
\begin{center}
  \textbf{\large{Registration Details}}
  \begin{itemize}
  \item \textbf{Thesis title:} Search for Quark Compositeness in $\gamma$ $+$ jet final states in proton-proton collisions at $\sqrt{s}$ = 13 TeV with the CMS
    Detector at the Large Hadron Collider.
  \item \textbf{Research Scholar:} Rocky Bala Garg
  \item \textbf{Thesis Supervisor:} Prof. Brajesh C. Choudhary
  \item \textbf{Department of Affiliation:} Department of Physics $\&$ Astrophysics
  \item \textbf{Ph.D. Registration No:} SF-I/Ph.D./488
   \item \textbf{Date of Registration:} October 18, 2011
  \end{itemize}
\end{center}




\end{document}
